\documentclass[11pt,reqno]{book}

\input{structure.tex} % Aqui estão todas as bibliotecas e definições

%---------------------------------------------------------------------
% Início do Documento  
\begin{document}

\renewcommand{\contentsname}{Sumário}
\renewcommand\indexname{Índice}


%-----------------------------------------------------------------------
%	CAPA
%-----------------------------------------------------------------------

% Entrar no arquivo "capa.tex" e modificar "Nome da Disciplina"

\begingroup

\thispagestyle{empty} % Sem Cabeçalho e Rodapé

\begin{tikzpicture}[remember picture,overlay]
\node[inner sep=0pt] (background) at (current page.center) {\includegraphics[width=\paperwidth]{Pictures/background.pdf}};
\draw (current page.center) node [fill=ocre!30!white,fill opacity=0.6,text opacity=1,inner sep=1cm]{\color{chapterhead}\Huge\centering\bfseries\sffamily\parbox[c][][t]{\paperwidth}{\centering Nome da Disciplina\\[15pt] % Nome da Matéria
{\Large Apostila Didática}\\[20pt]
% {\huge Prof. Nome do Professor }
}};
\end{tikzpicture}

\vfill

\endgroup




%---------------------------------------------------------------------
%	Página de Apresentação 
%---------------------------------------------------------------------

% Entrar no arquivo "aparesentacao.tex" e redigir o texto de apresentação da disciplina

\cleardoublepage
\newpage
~\vfill
\thispagestyle{empty}
\doublespacing 
\input{apresentacao}
\vspace{3cm}
\DTMlangsetup{showdayofmonth=false}
\noindent \textit{1ª Ed., \today }
\DTMlangsetup{showdayofmonth=true}

%---------------------------------------------------------------------
%	SUMÁRIO
%---------------------------------------------------------------------

% Não é necessário fazer modificação

\chapterimage{capitulo.png}
\pagestyle{empty}
\tableofcontents
\cleardoublepage
\pagestyle{fancy}


%-------------------------------------------------------------
%	INSTRUÇÕES
%-------------------------------------------------------------
%
% A partir deste ponto serão inseridos os capítulos da apostila através dos seguintes passos:
% 
% 1) Incluir um mini resumo de uma parte do documento com o comando "\parte{Nome da Parte}". (Opcional)
% 2) Adicionar uma nova pasta 
% 2) Incluir um capítulo com o comando "\input{"Caminho do Capítulo"}. Ex.: Caso queira incluir o capítulo 1 inserir o comando "\input{Capitulo 1/capitulo}".
% 
%-------------------------------------------------------------
%	DIVISÃO POR PARTES
%-------------------------------------------------------------

\parte{Parte Um}  % Comente essa linha (ctrl+/) se não quiser dividir a apostila em partes
% Título do capítulo
\capitulo{Escreva o título do capítulo aqui} \label{cap1}

  Aqui fica o texto introdutório do capítulo.
  Este é o texto de um capítulo.



\secao{Titulo da seção}\index{seção} % Adiciona  capítulo na parte 1. Para escrever o capítulo 1, abra o arquivo "/Capitulo 1/capitulo1.tex". O mesmo raciocínio vale para os demais.
% Título do capítulo
\capitulo{Escreva o título do capítulo aqui} \label{cap2}

  Aqui fica o texto introdutório do capítulo.
  Este é o texto de um capítulo.



\secao{Titulo da seção}\index{seção}


\parte{Parte Dois}  % Comente essa linha (ctrl+/) se não quiser dividir a apostila em partes
% Título do capítulo
\capitulo{Escreva o título do capítulo aqui}\label{cap3}

  Aqui fica o texto introdutório do capítulo.
  Este é o texto de um capítulo.



\secao{Titulo da seção}\index{seção} % Adicionar  capítulo na parte 2
%% Título do capítulo
\capitulo{Escreva o título do capítulo aqui}\label{cap4}

  Aqui fica o texto introdutório do capítulo.
  Este é o texto de um capítulo.



\secao{Titulo da seção}\index{seção}

% Capítulo com exemplos de comandos, comente quando quiser esconder no seu texto
\input{capitulo exemplos/exemplos} 

%-----------------------------------------------------------------
%	BIBLIOGRAFIA
%-----------------------------------------------------------------

\capitulo*{Bibliografia}
\addcontentsline{toc}{chapter}{\textcolor{ocre}{Bibliografia}} \nocite{*}
% Adiciona a Bibliografia para o Sumário

%------------------------------------------------
%   Artigos
%------------------------------------------------

\secao*{Artigos}
\addcontentsline{toc}{section}{Artigos}
\printbibliography[heading=bibempty,type=article]

%------------------------------------------------
%   Livros
%------------------------------------------------

\secao*{Livros}
\addcontentsline{toc}{section}{Livros}
\printbibliography[heading=bibempty,type=book]

%------------------------------------------------
%   Manuais
%------------------------------------------------

\secao*{Manuais}
\addcontentsline{toc}{section}{Manuais}
\printbibliography[heading=bibempty,type=manual]

%------------------------------------------------
%   Sites
%------------------------------------------------

\secao*{Sites}
\addcontentsline{toc}{section}{Sites}
\printbibliography[heading=bibempty,type=misc]


%---------------------------------------------------------------
%	ÍNDICE
%---------------------------------------------------------------

\cleardoublepage 
\phantomsection
\setlength{\columnsep}{0.75cm} 
\addcontentsline{toc}{chapter}{\textcolor{ocre}{Índice}} 
\printindex 
%---------------------------------------------------------------

\end{document}

%Fim do Documento 